\begin{frame}{检索增强生成 (Retrieval-Augmented Generation)}
    \begin{itemize}
        \item \textbf{核心痛点 (Pain Points):}
        \begin{itemize}
            \item \textbf{幻觉 (Hallucination):} 模型可能会一本正经地胡说八道。
            \item \textbf{知识时效性:} 模型的知识截止于训练结束那一刻,无法回答实时问题。
        \end{itemize}

        \item \textbf{工作流程 (Workflow):}
        \begin{enumerate}
            \item \textbf{Retrieve (检索):} 将用户的问题向量化,在外部向量数据库中匹配最相关的文档片段。
            \item \textbf{Augment (增强):} 将检索到的事实作为上下文 (Context) 拼接到 Prompt 中。
            \item \textbf{Generate (生成):} LLM 基于增强后的 Prompt 生成准确且可溯源的答案。
        \end{enumerate}
    \end{itemize}
\end{frame}

\begin{frame}{多智能体系统 (Multi-Agent Systems)}
    \begin{itemize}
        \item \textbf{从单体到群体 (From Solo to Swarm):}
        面对复杂的长流程任务(如开发一个游戏),单个 LLM 容易遗忘上下文或逻辑混乱。

        \item \textbf{协作机制 (Collaboration Mechanism):}
        \begin{itemize}
            \item \textbf{角色扮演 (Role Playing):} 不同的 Agent 扮演不同角色(如产品经理、架构师、工程师、测试员)。
            \item \textbf{流程标准化 (SOP):} 将复杂任务拆解为流水线,Agent 之间通过对话进行规划、执行和互相审查。
        \end{itemize}

        \item \textbf{典型框架:}
        \begin{itemize}
            \item \textbf{AutoGen (Microsoft):} 允许创建可定制、可对话的 Agent 网络。
            \item \textbf{MetaGPT:} 只要给一句需求,就能输出整个软件工程项目
        \end{itemize}
    \end{itemize}
\end{frame}

\begin{frame}{推理加速关键技术:KV Cache}
    \begin{itemize}
        \item \textbf{背景 (Background):}
        LLM 推理是自回归 (Auto-regressive) 的,每生成一个 Token,都要重新计算一次 Attention。

        \item \textbf{空间换时间 (Space-Time Trade-off):}
        \begin{itemize}
            \item Attention 公式: $\text{Attention}(Q, K, V) = \text{softmax}(\frac{QK^T}{\sqrt{d_k}})V$
						\item 生成第 $t$ 个 token 时,前 $t-1$ 个 token 的 $K$ (Key) 和 $V$ (Value) 矩阵计算结果是\textbf{不变的}。
            \item \textbf{KV Cache:} 将历史的 $K, V$ 向量缓存在显存中,每次只计算当前新 token 的 $Q, K, V$,避免重复计算。
        \end{itemize}

        \item \textbf{挑战 (Challenge):}
        \begin{itemize}
            \item \textbf{显存杀手:} KV Cache 随序列长度线性增长,是长文本推理显存占用过高(Memory Bound)的主要原因。
            \item \textit{解决方案示例:PagedAttention (vLLM).}
        \end{itemize}
    \end{itemize}
\end{frame}
