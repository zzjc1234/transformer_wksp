\begin{frame}{Introduction}
	\begin{itemize}
		\item 自20世纪50年代图灵测试提出以来,人类一直在探索如何让机器掌握语言智能。
		\item 语言本质上是一个复杂精妙的人类表达系统,受语法规则的约束。
		\item 开发能够理解和掌握语言的人工智能(AI)算法是一项巨大的挑战
	\end{itemize}
\end{frame}

\begin{frame}{什么是大语言模型 (What are LLMs?)}
	\begin{itemize}
		\item 基于深度学习的大规模概率模型,参数量通常在数十亿 (Billions) 至万亿级别。
		\item 这里的“大”既指\textbf{参数量巨大},也指训练数据的\textbf{规模巨大}(万亿级 Tokens)。
		\item 绝大多数现代 LLM 基于 \textbf{Transformer} 架构(特别是 Decoder-only 结构,如 GPT 系列)
		\item 利用自注意力机制 (Self-Attention) 捕捉长距离的文本依赖关系。
		\item \textbf{自监督学习 (Self-Supervised Learning):} 不需要人工标注,通过“预测下一个词” (Next Token Prediction) 进行训练。
		\item \textbf{知识压缩 (Knowledge Compression):} 模型不仅仅是在记忆文本,而是将世界知识、逻辑规则和语言模式压缩进了神经网络的权重参数中。
	\end{itemize}
\end{frame}

\begin{frame}{什么是大模型}
	\begin{center}
		\includegraphics[width=0.7\textwidth]{intro/llm_hist.png}
	\end{center}
\end{frame}

\begin{frame}{Historical Timeline of LLMs}
	\begin{itemize}
		\item SLM: 统计语言模型
		\item NLM: 神经语言模型
		\item PLM: 预训练语言模型
		\item LLM: 大语言模型
	\end{itemize}
\end{frame}

\begin{frame}{Historical Timeline of LLMs}
	\begin{center}
		\includegraphics[width=0.6\textwidth]{intro/arch_hist.png}
	\end{center}
\end{frame}

\begin{frame}{从 PLM 到 LLM 的范式转变}
    \begin{itemize}
        \item PLM (Pre-trained Language Models) 时代 (e.g., BERT, RoBERTa):
        \begin{itemize}
            \item \textbf{范式:} 预训练 + 微调 (Pre-train + Fine-tune)。
            \item \textbf{局限:} 需要针对特定任务(Task-specific)收集标注数据并更新参数;模型往往是“专才”而非“通才”。
        \end{itemize}

        \item LLM (Large Language Models) 时代 (e.g., GPT-3, GPT-4, Llama):
        \begin{itemize}
            \item \textbf{范式:} 预训练 + 提示工程/上下文学习 (Pre-train + Prompt/In-Context Learning)。
            \item \textbf{优势:} 冻结参数即可处理多种任务;通过指令微调 (Instruction Tuning) 实现通用意图理解。
        \end{itemize}
    \end{itemize}
\end{frame}

\begin{frame}{LLM 的核心进步 (Key Advancements)}
    \begin{itemize}
        \item \textbf{扩展定律 (Scaling Laws):}
        \begin{itemize}
            \item 随着参数量、数据量和计算量的增加,模型性能呈幂律提升,并未出现瓶颈 \cite{kaplan2020scalinglawsneurallanguage}。
        \end{itemize}

        \item \textbf{指令微调 (Instruction Tuning):}
        \begin{itemize}
            \item 通过在多任务指令数据集上微调,模型学会了“遵循指令”而非仅仅“预测下一个词”,显著提高了零样本泛化能力 \cite{ouyang2022traininglanguagemodelsfollow}。
        \end{itemize}

        \item \textbf{人类反馈强化学习 (RLHF):}
        \begin{itemize}
            \item 将模型输出与人类价值观(有用性、诚实性、安全性)对齐,解决了PLM生成内容不可控的问题 \cite{ouyang2022traininglanguagemodelsfollow}。
        \end{itemize}
    \end{itemize}
\end{frame}

\begin{frame}{LLM 的特殊能力 (Emergent Abilities)}
    当模型规模突破一定临界值(如 $10^{22}$ FLOPs)时,会出现小模型不具备的\textbf{涌现能力 (Emergent Abilities)} \cite{wei2022emergentabilitieslargelanguage}:

    \begin{itemize}
        \item \textbf{上下文学习 (In-Context Learning, ICL):}
        \begin{itemize}
            \item 模型无需梯度更新,仅通过Prompt中的几个示例即可学会新任务 \cite{brown2020languagemodelsfewshotlearners}。
        \end{itemize}

        \item \textbf{思维链 (Chain-of-Thought, CoT):}
        \begin{itemize}
            \item 通过生成中间推理步骤,显著提升了处理复杂逻辑、数学和常识推理任务的能力 \cite{wei2023chainofthoughtpromptingelicitsreasoning}。
        \end{itemize}

        \item \textbf{指令遵循与泛化:}
        \begin{itemize}
            \item 能够处理训练集中从未见过的任务描述,表现出真正的通用智能特征。
        \end{itemize}
    \end{itemize}
\end{frame}

\begin{frame}{大语言模型的当前应用}
    \begin{itemize}
        \item \textbf{编程智能体:}
        GitHub Copilot, Cursor, CodeT5, 以及基于 GPT 的编程助手。

        \item \textbf{对话式 AI:}
        ChatGPT, Claude, Gemini, 用于客户服务及个人助理。

        \item \textbf{内容生成 (Content Generation):}
        自动化文章撰写、创意写作、营销内容生成。

        \item \textbf{科研辅助 (Research Assistance):}
        Alpha evolve, GPT Deepresearch, 文献综述加速、研究假设生成、辅助数据分析。
    \end{itemize}
\end{frame}

\begin{frame}{Attention Mechanism}
	\begin{center}
		\includegraphics[width=0.7\textwidth]{intro/atten_paper.png}
	\end{center}
\end{frame}

\begin{frame}{Building LLM}
	\begin{center}
		\includegraphics[width=0.8\textwidth]{intro/build_llm.png}
	\end{center}
\end{frame}
